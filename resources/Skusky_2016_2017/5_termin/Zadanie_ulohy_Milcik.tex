\documentclass[12pt]{article}
\usepackage{graphics,amssymb,amsmath}

\usepackage[slovak]{babel} 
\usepackage[cp1250]{inputenc}
\usepackage[IL2]{fontenc}

\usepackage{multicol}
\usepackage{mathtools}

\pagestyle{empty}
\setlength\textwidth{170mm}
\setlength\textheight{265mm}
\addtolength\oddsidemargin{-20mm}
\addtolength\topmargin{-20mm}
\setlength{\parindent}{1pt}
\setlength{\parskip}{10pt}
\newcount\pocet
\pocet = 1
\def\pr{{\bf \the \pocet .\ \global\advance\pocet by 1}}

\newcommand{\g}{ \dots \dots \dots \dots \dots \ }
\newcommand{\gu}{ \dots \dots \ }
\newcommand{\gr}{\dotfill \ }

\begin{document}

\pr  Dan� je funkcia $f(x,y)=x^3+3xy^2-51x-24y$ a oblas� $M$. \\
Oblas� $M$ je mnohouholn�k $ABCD$, ktor�ho vrcholy maj� s�radnice $A=[0,0]$, $B=[6,0]$, $C=[2,2]$ a $D=[6,4]$.
\begin{enumerate}
\item[a)] Na�rtnite oblas� $M$: 

\textbf{N��rt:}
\vspace{5cm}

\textbf{Pomocou rovn�c pop�te hranice oblasti $M$:}
\begin{enumerate}
\item $AB$ \gr
\item $AC$ \gr
\item $BD$ \gr
\item $CD$ \gr
\end{enumerate}

\item[b)]N�jdite lok�lne extr�my funkcie v $M$.\\
\hspace{5cm} \textbf{Dopl�te odpove�:}
Funkcia $f(x,y)$ m� v bode \gr lok�lne \gr
\medskip

\item[c)] N�jdite viazan� lok�lne extr�my funkcie na hraniciach oblasti $M$.\\
Na hranici
\begin{enumerate}
\item $AB$ m� funkcia $f(x,y)$ m� v bode \gr viazan� lok�lne \gr
\item $AC$ m� funkcia $f(x,y)$ m� v bode \gr viazan� lok�lne \gr
\item $BD$ m� funkcia $f(x,y)$ m� v bode \gr viazan� lok�lne \gr
\item $CD$ m� funkcia $f(x,y)$ m� v bode \gr viazan� lok�lne \gr
\end{enumerate}
\item[d)] N�jdite najv��iu a najmen�iu hodnotu funkcie $f(x,y)$ na oblasti $M$.\\

\textbf{Najv��ia} hodnota funkcie $f(x,y)$ je: \gr

\textbf{Najmen�ia} hodnota funkcie $f(x,y)$ je: \gr

\end{enumerate}

\end{document}
