\documentclass[12pt]{article}
\usepackage{graphics,amssymb,amsmath}

\usepackage[slovak]{babel} 
\usepackage[cp1250]{inputenc}
\usepackage[IL2]{fontenc}

\pagestyle{empty}
\setlength\textwidth{170mm}
\setlength\textheight{265mm}
\addtolength\oddsidemargin{-20mm}
\addtolength\topmargin{-25mm}
\setlength{\parindent}{1pt}
\setlength{\parskip}{10pt}
\newcount\pocet
\pocet = 1
\def\pr{{\bf \the \pocet .\ \global\advance\pocet by 1}}

\newcommand{\g}{ \dots \dots \dots \dots \dots \ }
\newcommand{\gu}{ \dots\dots\dots }
\newcommand{\gr}{\dotfill \ }

\begin{document}
\section{Matematika I}
\subsection{Projekt - zadanie �loh}
\pr Zobrazte v rovine defini�n� obor funkcie $$f(x,y)=\ln(y-x)+\ln(x+y).$$

\pr N�jdite lok�lne extr�my funkcie $$f(x,y)=x^2+y^2+3xy+2.$$

\pr Ur�te viazan� extr�my funkcie $$f(x,y)=x^2+y^2+3xy+2$$ na hraniciach oblasti $M$, ktor� je obmedzen� grafmi funkci� $y=\vert x\vert$ a $y=2$.

\pr N�jdite v�eobecn� rie�enie diferenci�lnej rovnice $$y^{\prime\prime}+y^{\prime}=q(x),$$
ak
\begin{enumerate}
\item[a)] $g(x)=2x^2-3$.
\item[b)] $g(x)=e^{-2x}$.
\item[c)] $g(x)=4xe^{x}$.
\item[d)] $g(x)=2\cos {2x}-\sin{2x}$.
\end{enumerate}
\end{document}
