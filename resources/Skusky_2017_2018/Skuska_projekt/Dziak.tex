\documentclass[12pt]{article}
\usepackage{graphics,amssymb,amsmath}

\usepackage[slovak]{babel} 
\usepackage[cp1250]{inputenc}
\usepackage[IL2]{fontenc}

\pagestyle{empty}
\setlength\textwidth{170mm}
\setlength\textheight{265mm}
\addtolength\oddsidemargin{-20mm}
\addtolength\topmargin{-25mm}
\setlength{\parindent}{1pt}
\setlength{\parskip}{10pt}
\newcount\pocet
\pocet = 1
\def\pr{{\bf \the \pocet .\ \global\advance\pocet by 1}}

\newcommand{\g}{ \dots \dots \dots \dots \dots \ }
\newcommand{\gu}{ \dots\dots\dots }
\newcommand{\gr}{\dotfill \ }

\begin{document}
\section{Matematika I}
\subsection{Projekt - zadanie �loh}
\pr Zobrazte v rovine defini�n� obor funkcie $$f(x,y)=\ln\left(\frac{x-y+1}{x^2+y^2-4}\right).$$

\pr N�jdite lok�lne extr�my funkcie $$f(x,y)=3x^2y+xy^2-6xy.$$

\pr Ur�te viazan� extr�my funkcie $$f(x,y)=2x^2+4y^2$$ na hraniciach oblasti $M$, ktor� je dan� $$x^2+y^2\leqq 9.$$

\pr Pomocou rezov jednotliv�mi rovinami zistite o ak� 3D �tvary ide:

\begin{enumerate}
\item[a)] $x\sp{2}-81y\sp{2}+9z\sp{2}-8x-162y-36z-110=0$.
\item[b)] $4y\sp{2}+x\sp{2}+z\sp{2}+12z+35=0$.
\end{enumerate}
\end{document}
