\documentclass[12pt]{article}
\usepackage{graphics,amssymb,amsmath}

\usepackage[slovak]{babel}
\usepackage[utf8]{inputenc}
\usepackage[IL2]{fontenc}

\usepackage{multicol}
\usepackage{mathtools}

\pagestyle{empty}
\setlength\textwidth{170mm}
\setlength\textheight{265mm}
\addtolength\oddsidemargin{-20mm}
\addtolength\topmargin{-20mm}
\setlength{\parindent}{1pt}
\setlength{\parskip}{10pt}
\newcount\pocet
\pocet = 1
\def\pr{{\bf \the \pocet .\ \global\advance\pocet by 1}}

\newcommand{\g}{ \dots \dots \dots \dots \dots \ }
\newcommand{\gu}{ \dots \dots \ }
\newcommand{\gr}{\dotfill \ }

\begin{document}

\newenvironment{itemize*}
 {\begin{itemize}
   \setlength{\itemsep}{0pt}
   \setlength{\parskip}{0pt}}
 {\end{itemize}}

\newenvironment{enumerate*}
 {\begin{enumerate}
   \setlength{\itemsep}{0pt}
   \setlength{\parskip}{0pt}}
 {\end{enumerate}}


\phantom{a}

\centerline{\textbf{\Large Matematika I}}


\vskip0.5cm

\centerline{\bf  Meno a priezvisko: \gr Podpis: \gr}
\vskip0.5cm
\centerline{\bf  Ročník: \gr Študijný program: \gr}
\vskip0.5cm

\medskip


\pr (6b) Daná je lineárna obyčajná diferenciálna rovnica (LODR)
$y^{\prime\prime}(x) +6y^{\prime}(x)= x$.

\begin{enumerate}
\item[a)] Napíšte charakteristickú rovnicu k danej diferenciálnej rovnici.

\item[b)] Nájdite fundamentálny systém riešení diferenciálnej rovnice s nulovou pravou stranou.

\item[c)] Nájdite partikulárne riešenie uvedenej nehomogénnej rovnice.

\item[c)] Napíšte všeobecné riešenie danej lineárnej diferenciálnej rovnice.

\end{enumerate}

\pr (4b) Daná je lineárna obyčajná diferenciálna rovnica (LODR)
$y^{\prime}(x) +6y(x)= e^{x}$.

\begin{enumerate}
\item[a)] Napíšte charakteristickú rovnicu k danej diferenciálnej rovnici.

\item[b)] Nájdite fundamentálny systém riešení diferenciálnej rovnice s nulovou pravou stranou.

\item[c)] Nájdite partikulárne riešenie uvedenej nehomogénnej rovnice.

\item[c)] Napíšte všeobecné riešenie danej lineárnej diferenciálnej rovnice.

\end{enumerate}

\newpage

\pr (20b) Daná je funkcia 
$f(x,y) = x^2+y^2-xy-x-y+2$ 
a oblasť $M$. \\
Oblasť $M$ je mnohouholník $ABCD$ s vrcholmi  
$A=[0,0]$, $B=[4,0]$, $C=[3,3]$ a $D=[0,3]$.

\begin{enumerate}
\item[a)] Načrtnite oblasť $M$:

\textbf{Náčrt:}
\vspace{5cm}

\textbf{Pomocou matematických vzťahov popíšte hranice oblasti $M$:}
\begin{enumerate}
\item $AB$ \gr
\item $BC$ \gr
\item $CD$ \gr
\item $AD$ \gr
\end{enumerate}

\item[b)] Nájdite lokálne extrémy danej funkcie $f(x, y)$ v oblasti $M$. \\
Ak hľadané lokálne extrémy nie sú, napíšte \uv{nie sú}.\\[1ex]
\hspace{5cm} \textbf{Doplňte odpoveď:}
Funkcia $f(x,y)$ má v bode \gr lokálne \gr
\medskip

\item[c)] Nájdite viazané lokálne extrémy danej funkcie $f(x, y)$ na hraniciach oblasti $M$.  
Ak hľadaný lokálny extrém nejestvuje, napíšte \uv{nie je}.

\begin{enumerate}
\item Na hranici $AB$ má funkcia $f(x,y)$  v bode \gr viazané lokálne \gr
\item Na hranici $BC$ má funkcia $f(x,y)$  v bode \gr viazané lokálne \gr
\item Na hranici $CD$ má funkcia $f(x,y)$  v bode \gr viazané lokálne \gr
\item Na hranici $AD$ má funkcia $f(x,y)$  v bode \gr viazané lokálne \gr
\end{enumerate}
\item[d)] Nájdite najväčšiu a najmenšiu hodnotu funkcie $f(x,y)$ na oblasti $M$.\\

\textbf{Najväčšia} hodnota funkcie $f(x,y)$ je: \gr

\textbf{Najmenšia} hodnota funkcie $f(x,y)$ je: \gr

\end{enumerate}

\end{document}
