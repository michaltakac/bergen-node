\documentclass[12pt]{article}
\usepackage{graphics,amssymb,amsmath}

\usepackage[slovak]{babel}
\usepackage[utf8]{inputenc}
\usepackage[IL2]{fontenc}

\usepackage{multicol}
\usepackage{mathtools}

\pagestyle{empty}
\setlength\textwidth{170mm}
\setlength\textheight{265mm}
\addtolength\oddsidemargin{-20mm}
\addtolength\topmargin{-20mm}
\setlength{\parindent}{1pt}
\setlength{\parskip}{10pt}
\newcount\pocet
\pocet = 1
\def\pr{{\bf \the \pocet .\ \global\advance\pocet by 1}}

\newcommand{\g}{ \dots \dots \dots \dots \dots \ }
\newcommand{\gu}{ \dots \dots \ }
\newcommand{\gr}{\dotfill \ }

\begin{document}

\newenvironment{itemize*}
 {\begin{itemize}
   \setlength{\itemsep}{0pt}
   \setlength{\parskip}{0pt}}
 {\end{itemize}}

\newenvironment{enumerate*}
 {\begin{enumerate}
   \setlength{\itemsep}{0pt}
   \setlength{\parskip}{0pt}}
 {\end{enumerate}}

\pr (6b) Daná je funkcia 
$\displaystyle f(x,y)=\frac{1}{x+y^2}$, 
bod $\displaystyle A=[1, \, 2]$ 
a vektor $\displaystyle \vec{l}=\left(0, \, 2\right)$.

\begin{enumerate}
\item[a)] (3b) Nájdite gradient funkcie $f(x,y)$ v bode $A$.
\medskip

\textbf{Gradient} funkcie $f(x,y)$ v bode $A$ je \gr

\item[b)] (3b) Vypočítajte deriváciu funkcie $f(x,y)$ v bode $A$ v smere vektora $\vec{l}$.
\medskip

\textbf{Derivácia} funkcie $f(x,y)$ v bode $A$ v smere vektora $\vec{l}$ je \gr
\end{enumerate}

\end{document}
