\documentclass[12pt]{article}
\usepackage{graphics,amssymb,amsmath}

\usepackage[slovak]{babel}
\usepackage[utf8]{inputenc}
\usepackage[IL2]{fontenc}

\usepackage{multicol}
\usepackage{mathtools}

\pagestyle{empty}
\setlength\textwidth{170mm}
\setlength\textheight{265mm}
\addtolength\oddsidemargin{-20mm}
\addtolength\topmargin{-20mm}
\setlength{\parindent}{1pt}
\setlength{\parskip}{10pt}
\newcount\pocet
\pocet = 1
\def\pr{{\bf \the \pocet .\ \global\advance\pocet by 1}}

\newcommand{\g}{ \dots \dots \dots \dots \dots \ }
\newcommand{\gu}{ \dots \dots \ }
\newcommand{\gr}{\dotfill \ }

\begin{document}

\newenvironment{itemize*}
 {\begin{itemize}
   \setlength{\itemsep}{0pt}
   \setlength{\parskip}{0pt}}
 {\end{itemize}}

\newenvironment{enumerate*}
 {\begin{enumerate}
   \setlength{\itemsep}{0pt}
   \setlength{\parskip}{0pt}}
 {\end{enumerate}}


\pr (8b) Daná je lineárna obyčajná diferenciálna rovnica (LODR)
$y^{\prime\prime}(x) +6y^{\prime}(x)= 3x$.

\begin{enumerate}
\item[a)](2b) Napíšte charakteristickú rovnicu k danej diferenciálnej rovnici.
\medskip

\textbf{Charakteristická rovnica je:} \gr

\item[b)] (2b) Nájdite fundamentálny systém riešení diferenciálnej rovnice s nulovou pravou stranou.

\medskip

\textbf{Fundamentálny systém riešení je} \gr

\item[b)] (2b) Nájdite partikulárne riešenie uvedenej nehomogénnej rovnice.

\medskip

\textbf{Partikulárne riešene je} \gr

\item[c)] (2b) Napíšte všeobecné riešenie danej lineárnej diferenciálnej rovnice.

\medskip

\textbf{Všeobecné riešenie danej LODR je} \gr
\end{enumerate}

\end{document}
