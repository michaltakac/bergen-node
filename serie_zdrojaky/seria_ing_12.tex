
\documentclass[12pt]{article}
\usepackage{graphics,amssymb,amsmath}

\usepackage[slovak]{babel}
\usepackage[utf8]{inputenc}
\usepackage[IL2]{fontenc}

\usepackage{multicol}
\usepackage{mathtools}

\pagestyle{empty}
\setlength\textwidth{170mm}
\setlength\textheight{265mm}
\addtolength\oddsidemargin{-20mm}
\addtolength\topmargin{-20mm}
\setlength{\parindent}{1pt}
\setlength{\parskip}{10pt}
\newcount\pocet
\pocet = 1
\def\pr{{\bf \the \pocet .\ \global\advance\pocet by 1}}

\newcommand{\g}{ \dots \dots \dots \dots \dots \ }
\newcommand{\gu}{ \dots \dots \ }
\newcommand{\gr}{\dotfill \ }

\begin{document}

\newenvironment{itemize*}
{\begin{itemize}
\setlength{\itemsep}{0pt}
\setlength{\parskip}{0pt}}
{\end{itemize}}

\newenvironment{enumerate*}
{\begin{enumerate}
\setlength{\itemsep}{0pt}
\setlength{\parskip}{0pt}}
{\end{enumerate}}


\phantom{a}

\centerline{\textbf{\Large Matematika I}}
\smallskip
\centerline{Séria úloh 12}
\vskip0.5cm

\medskip

\pr (11b) Daná je všeobecná rovnica kužeľosečky $4x^2-y^2-24x+4y+28=0$.\\
\textbf{Doplňte}
\begin{enumerate}
\item[a)] (2b) Stredová rovnica kužeľosečky je\gr
\item[b)] (1b) Kužeľosečka je \gr typu.
\item[c)] (3b) Popíšte (ak existujú):
\begin{enumerate}
\item[$c_1$)] dĺžka hlavnej poloosi je \gr
\item[$c_2$)] dĺžka vedľajšej poloosi je \gr
\item[$c_3$)] excentricita je \gr
\end{enumerate}
\item[d)] (4b) Napíšte súradnice (ak existujú):
\begin{itemize}
\item[$d_1$)] stredu kužeľosečky \gr
\item[$d_2$)] hlavných vrcholov kužeľosečky \gr
\item[$d_3$)] vedľajších vrcholov kužeľosečky \gr
\item[$d_4$)] súradnice ohniska resp. ohnísk kužeľosečky \gr
\end{itemize}
\item[e)] (1b) Znázornite kužeľosečku a v náčrte popíšte jej významné prvky.
\end{enumerate}

\newpage

\pr (2b) Vyberte funkciu, ktorej definičný obor je znázornený na obrázku.
\begin{multicols}{2}
\noindent
\includegraphics[width=6cm]{kruznica4.pdf}
\noindent
\begin{itemize}
\item[a)] $\displaystyle f(x,y)= \frac{\ln{(x^2+y^2-1)}}{\sqrt{4-x^2-y^2}}$
\item[b)] $\displaystyle f(x,y)= \frac{\ln{(4-x^2-y^2)}}{\sqrt{x^2+y^2-1}}$
\item[c)] $\displaystyle f(x,y)= \frac{\sqrt{x^2+y^2-1}}{\ln{(4-x^2-y^2)}}$
\item[d)] $\displaystyle f(x,y)= \frac{\sqrt{4-x^2-y^2}}{\ln(x^2+y^2-1)}$
\end{itemize}
\end{multicols}

\pr (6b) Vypočítajte
$$\iint\limits_{M}{x} \ \mathrm{d}x \mathrm{d}y,$$
kde množina $M$ je mnohouholník s vrcholmi
$A=[-1, -1]$,
$B=[1, -1]$,
$C=[2, 2]$,
$D=[-2, 2]$.

\begin{enumerate}
\item[]\textbf{Výsledok:}\gr
\end{enumerate}

\pr (4b) Toto je príklad typu G

text text text \\ \\

\pr (8b) Daná je lineárna obyčajná diferenciálna rovnica (LODR)
$y^{\prime}(x)+y(x)= x+1$.

\begin{enumerate}
\item[a)](2b) Napíšte charakteristickú rovnicu k danej diferenciálnej rovnici.
\medskip

\textbf{Charakteristická rovnica je:} \gr

\item[b)] (2b) Nájdite fundamentálny systém riešení diferenciálnej rovnice s nulovou pravou stranou.

\medskip

\textbf{Fundamentálny systém riešení je} \gr

\item[b)] (2b) Nájdite partikulárne riešenie uvedenej nehomogénnej rovnice.

\medskip

\textbf{Partikulárne riešene je} \gr

\item[c)] (2b) Napíšte všeobecné riešenie danej lineárnej diferenciálnej rovnice.

\medskip

\textbf{Všeobecné riešenie danej LODR je} \gr
\end{enumerate}

\pr (4b)
Vypočítajte, ak existuje

$$
\lim\limits_{
[x,y]\rightarrow [1,2]}
\frac{2-\sqrt{4-xy}}{xy}.
$$

\begin{enumerate}
\item[]\textbf{Výsledok:}\gr
\end{enumerate}

\pr (6b) Nájdite rovnicu dotykovej roviny $\tau$ ku grafu funkcie
$\displaystyle f(x,y)=\sqrt{14-x^2-y^2}$ \\
\hspace*{1.4cm} v bode $T=\left[3,1, z_{0}\right]$.

\begin{enumerate}
\item[] (2b) Nájdite $z_0$ a  \textbf{uveďte súradnice dotykového bodu}: \gr
\item[] (4b) Všeobecná \textbf{rovnica} dotykovej roviny $\tau$ je:\gr
\end{enumerate}

\pr (6b) Daná je funkcia
$\displaystyle f(x,y)=3x^4-x^2y^3+y^2$,
bod $\displaystyle A=[1, \,-1 ]$
a vektor $\displaystyle \vec{l}=\left(-1, \, 2\right)$.

\begin{enumerate}
\item[a)] (3b) Nájdite gradient funkcie $f(x,y)$ v bode $A$.
\medskip

\textbf{Gradient} funkcie $f(x,y)$ v bode $A$ je \gr

\item[b)] (3b) Vypočítajte deriváciu funkcie $f(x,y)$ v bode $A$ v smere vektora $\vec{l}$.
\medskip

\textbf{Derivácia} funkcie $f(x,y)$ v bode $A$ v smere vektora $\vec{l}$ je \gr
\end{enumerate}

\pr (9b) Toto je príklad typu E

text text text \\ \\


\end{document}