
\documentclass[12pt]{article}
\usepackage{graphics,amssymb,amsmath}

\usepackage[slovak]{babel}
\usepackage[utf8]{inputenc}
\usepackage[IL2]{fontenc}

\usepackage{multicol}
\usepackage{mathtools}

\pagestyle{empty}
\setlength\textwidth{170mm}
\setlength\textheight{265mm}
\addtolength\oddsidemargin{-20mm}
\addtolength\topmargin{-20mm}
\setlength{\parindent}{1pt}
\setlength{\parskip}{10pt}
\newcount\pocet
\pocet = 1
\def\pr{{\bf \the \pocet .\ \global\advance\pocet by 1}}

\newcommand{\g}{ \dots \dots \dots \dots \dots \ }
\newcommand{\gu}{ \dots \dots \ }
\newcommand{\gr}{\dotfill \ }

\begin{document}

\newenvironment{itemize*}
{\begin{itemize}
\setlength{\itemsep}{0pt}
\setlength{\parskip}{0pt}}
{\end{itemize}}

\newenvironment{enumerate*}
{\begin{enumerate}
\setlength{\itemsep}{0pt}
\setlength{\parskip}{0pt}}
{\end{enumerate}}


\phantom{a}

\centerline{\textbf{\Large Matematika I}}
\smallskip
\centerline{Séria úloh 21}
\vskip0.5cm

\medskip

\pr (11b) Daná je všeobecná rovnica kužeľosečky $4x^2-y^2-24x+4y+28=0$.\\
\textbf{Doplňte}
\begin{enumerate}
\item[a)] (2b) Stredová rovnica kužeľosečky je\gr
\item[b)] (1b) Kužeľosečka je \gr typu.
\item[c)] (3b) Popíšte (ak existujú):
\begin{enumerate}
\item[$c_1$)] dĺžka hlavnej poloosi je \gr
\item[$c_2$)] dĺžka vedľajšej poloosi je \gr
\item[$c_3$)] excentricita je \gr
\end{enumerate}
\item[d)] (4b) Napíšte súradnice (ak existujú):
\begin{itemize}
\item[$d_1$)] stredu kužeľosečky \gr
\item[$d_2$)] hlavných vrcholov kužeľosečky \gr
\item[$d_3$)] vedľajších vrcholov kužeľosečky \gr
\item[$d_4$)] súradnice ohniska resp. ohnísk kužeľosečky \gr
\end{itemize}
\item[e)] (1b) Znázornite kužeľosečku a v náčrte popíšte jej významné prvky.
\end{enumerate}

\newpage

\pr (2b) Vyberte funkciu, ktorej definičný obor je znázornený na obrázku.
\begin{multicols}{2}
\noindent
\includegraphics[width=6cm]{kruznica4.pdf}
\noindent
\begin{itemize}
\item[a)] $\displaystyle f(x,y)= \frac{\ln{(x^2+y^2-1)}}{\sqrt{4-x^2-y^2}}$
\item[b)] $\displaystyle f(x,y)= \frac{\sqrt{4-x^2-y^2}}{\ln(x^2+y^2-1)}$
\item[c)] $\displaystyle f(x,y)= \frac{\ln{(4-x^2-y^2)}}{\sqrt{x^2+y^2-1}}$
\item[d)] $\displaystyle f(x,y)= \frac{\sqrt{x^2+y^2-1}}{\ln{(4-x^2-y^2)}}$
\end{itemize}
\end{multicols}

\pr (6b) Vypočítajte $$\displaystyle \iint\limits_{M}{x^2y} \ \mathrm{d}x \mathrm{d}y,$$ kde množina $M$ je obdĺžnik s vrcholmi $A=[1,1]$, $B=[2,1]$, $C=[2,2]$ a $D=[1,2]$.

\begin{enumerate}
\item[]\textbf{Výsledok:}\gr
\end{enumerate}

\pr (4b) Bod $M$ má v pravouhlej súradnicovej sústave súradnice:  $M=[3,\sqrt{3}, 3]$.

\begin{enumerate}
\item[a)] (2b) Vyberte správnu odpoveď:\\
Súradnice bodu $M$ v cylindrickej súradnicovej sústave sú:


\noindent
\begin{itemize}
\begin{multicols}{2}
\item[a)]  $\displaystyle M=[2\sqrt{3}, \, -\frac{\pi}{6}, \, 3]$
\item[b)]  $\displaystyle M=[2\sqrt{3}, \, -\frac{\pi}{3},  \, 3]$

\item[c)]  $\displaystyle M=[2\sqrt{3}, \, \frac{\pi}{3},  \, 3]$
\item[d)]  $\displaystyle M=[2\sqrt{3}, \, \frac{\pi}{6},  \, 3]$
\end{multicols}
\end{itemize}


\item[b)] (2b) Znázornite tento bod $M$ v cylindrickej súradnicovej sústave.

\begin{enumerate}
\item[]\textbf{Náčrt:}
\end{enumerate}
\end{enumerate}

\newpage

\bigskip

\pr (8b) Daná je lineárna obyčajná diferenciálna rovnica (LODR)
$y^{\prime\prime}(x)+y(x)= 3/sin x$.

\begin{enumerate}
\item[a)](2b) Napíšte charakteristickú rovnicu k danej diferenciálnej rovnici.
\medskip

\textbf{Charakteristická rovnica je:} \gr

\item[b)] (2b) Nájdite fundamentálny systém riešení diferenciálnej rovnice s nulovou pravou stranou.

\medskip

\textbf{Fundamentálny systém riešení je} \gr

\item[c)] (2b) Nájdite partikulárne riešenie uvedenej nehomogénnej rovnice.

\medskip

\textbf{Partikulárne riešene je} \gr

\item[d)] (2b) Napíšte všeobecné riešenie danej lineárnej diferenciálnej rovnice.

\medskip

\textbf{Všeobecné riešenie danej LODR je} \gr
\end{enumerate}

\pr (4b) Vypočítajte nasledujúcu limitu

$$
\lim\limits_{
[x,y]\rightarrow [0,\frac{\pi}{2}]}
\frac{\sin(x+y)}{x+y}.
$$
\begin{enumerate}
\item[]\textbf{Výsledok:}\gr
\end{enumerate}

\pr (6b) Napíšte všeobecnú rovnicu dotykovej roviny ku grafu funkcie $$f(x,y)=x^2+y^2-4x+6y+8,$$ ak hľadaná dotyková rovina je rovnobežná s rovinou $R_{xy}$.

\begin{enumerate}

\item[]\textbf{Súradnice dotykového bodu sú:} \gr
\item[]\textbf{Všeobecná rovnica dotykovej roviny je:} \gr
\end{enumerate}

\pr (6b) Daná je funkcia
$\displaystyle f(x,y)=\sqrt{x^2-y^2}$,
bod $\displaystyle A=[5, \,3 ]$
a vektor $\displaystyle \vec{l}=\left(-1, \, 2\right)$.

\begin{enumerate}
\item[a)] (3b) Nájdite gradient funkcie $f(x,y)$ v bode $A$.
\medskip

\textbf{Gradient} funkcie $f(x,y)$ v bode $A$ je \gr

\item[b)] (3b) Vypočítajte deriváciu funkcie $f(x,y)$ v bode $A$ v smere vektora $\vec{l}$.
\medskip

\textbf{Derivácia} funkcie $f(x,y)$ v bode $A$ v smere vektora $\vec{l}$ je \gr
\end{enumerate}

\newpage

\bigskip

\pr (27b) Daná je funkcia
$f(x,y) = x^3+y^3-9xy+27$
a oblasť $M$. \\
Oblasť $M$ je mnohouholník $ABCD$ s vrcholmi
$A=[0,0]$, $B=[4,0]$, $C=[4,4]$ a $D=[0,4]$.

\begin{enumerate}
\item[a)] Načrtnite oblasť $M$:

\textbf{Náčrt:}
\vspace{5cm}

\textbf{Pomocou matematických vzťahov popíšte hranice oblasti $M$:}
\begin{enumerate}
\item (2b) $AB$ \gr
\item (2b) $BC$ \gr
\item (2b) $CD$ \gr
\item (2b) $AD$ \gr
\end{enumerate}

\item[b)](5b) Nájdite lokálne extrémy danej funkcie $f(x, y)$ v oblasti $M$. \\
Ak hľadané lokálne extrémy nie sú, napíšte \uv{nie sú}.\\[1ex]
\hspace{5cm} \textbf{Doplňte odpoveď:}
Funkcia $f(x,y)$ má v bode \gr lokálne \gr
\medskip

\item[c)] Nájdite viazané lokálne extrémy danej funkcie $f(x, y)$ na hraniciach oblasti $M$.
Ak hľadaný lokálny extrém nejestvuje, napíšte \uv{nie je}.

\begin{enumerate}
\item (3b) Na hranici $AB$ má funkcia $f(x,y)$  v bode \gr viazané lokálne \gr
\item (3b) Na hranici $BC$ má funkcia $f(x,y)$  v bode \gr viazané lokálne \gr
\item (3b) Na hranici $CD$ má funkcia $f(x,y)$  v bode \gr viazané lokálne \gr
\item (3b) Na hranici $AD$ má funkcia $f(x,y)$  v bode \gr viazané lokálne \gr
\end{enumerate}
\item[d)] (2b) Nájdite najväčšiu a najmenšiu hodnotu funkcie $f(x,y)$ na oblasti $M$.\\

\textbf{Najväčšia} hodnota funkcie $f(x,y)$ je: \gr

\textbf{Najmenšia} hodnota funkcie $f(x,y)$ je: \gr

\end{enumerate}


\end{document}